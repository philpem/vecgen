% A command that's useful for saving typing
\newcommand{\opcodeentry}[3]{
	\addcontentsline{toc}{subsection}{#1 --- #2}
	\index{Opcodes!#1 (#2)}
	\index{Mnemonics!#2 (#1)}
	\begin{center}
		\begin{tabular*}{1\textwidth}{@{\extracolsep{\fill}} | l r | }
			\hline
				& \\[-7pt]
				\multirow{2}{*}{\huge\bf #1}			&	{\large\bf #3}	\\
														&	{\large\bf #2}	\\
			\hline
	   	\end{tabular*}
	\end{center}
}

\newcommand{\mcol}[1]{
	\multicolumn{1}{c}{#1}
}

\newcommand{\mcolb}[1]{
	\multicolumn{1}{|c|}{#1}
}

%%%%%%%%%%%%%%%%%%%%%%%%%%%%%%%%%%%%%%%%%%%%%%%%%%%%%%%%%%%%%%%%%%%%%%%%%%%%%%%
%%%%%%%%%%%%%%%%%%%%%%%%%%%%%%%%%%%%%%%%%%%%%%%%%%%%%%%%%%%%%%%%%%%%%%%%%%%%%%%

% Save parindent
\newlength{\savedparindent}
\setlength{\savedparindent}{\parindent}

% Switch off parindent
\setlength{\parindent}{0mm}

%%%%%%%%%%%%%%%%%%%%%%%%%%%%%%%%%%%%%%%%%%%%%%%%%%%%%%%%%%%%%%%%%%%%%%%%%%%%%%%
%%%%%%%%%%%%%%%%%%%%%%%%%%%%%%%%%%%%%%%%%%%%%%%%%%%%%%%%%%%%%%%%%%%%%%%%%%%%%%%

\section{Opcodes}

%%%%%%%%%%%%%%%%%%%%%%%%%%%%%%%%%%%%%%%%%%%%%%%%%%%%%%%%%%%%%%%%%%%%%%%%%%%%%%%
%% VCTR: 00h-09h

\begin{minipage}{\textwidth}   \setlength{\parindent}{\savedparindent}
\opcodeentry{VCTR}{Draw Long Vector}{Opcodes 00h -- 09h}
%{\bf Encoding:}
\begin{tabular*}{1\textwidth}{@{\extracolsep{\fill}} |c|c|c|c|c|c|c|c|c|c|c|c|c|c|c|c|}
	\multicolumn{16}{ c }{\bf Word 0} \\
	\mcol{15} & \mcol{14} & \mcol{13} & \mcol{12} & \mcol{11} & \mcol{10} & \mcol{9} & \mcol{8} & \mcol{7} & \mcol{6} & \mcol{5} & \mcol{4} & \mcol{3} & \mcol{2} & \mcol{1} & \mcol{0} \\ \hline
	\multicolumn{4}{|c|}{OP[3..0]}	& 0		& Y\tsub{S}	& \multicolumn{10}{c|}{Y[9..0]} \\ \hline
\end{tabular*}
\\
\vspace{2 mm}
\\
\begin{tabular*}{1\textwidth}{@{\extracolsep{\fill}} |c|c|c|c|c|c|c|c|c|c|c|c|c|c|c|c|}
	\multicolumn{16}{ c }{\bf Word 1} \\
	\mcol{15} & \mcol{14} & \mcol{13} & \mcol{12} & \mcol{11} & \mcol{10} & \mcol{9} & \mcol{8} & \mcol{7} & \mcol{6} & \mcol{5} & \mcol{4} & \mcol{3} & \mcol{2} & \mcol{1} & \mcol{0} \\ \hline
	\multicolumn{4}{|c|}{Z[3..0]}	& 0		& X\tsub{S}	& \multicolumn{10}{c|}{X[9..0]} \\ \hline
\end{tabular*} \\ \\
%%
{\bf Operands:}

\begin{tabular}{ l l }
	{\bf OP[3..0]}	&	Opcode nibble.		\\
	{\bf Y\tsub{S}}	&	Y axis sign bit.	\\
	{\bf Y[9..0]}	&	Y axis position.	\\
	{\bf X\tsub{S}}	&	X axis sign bit.	\\
	{\bf X[9..0]}	&	X axis position.	\\
	{\bf Z[3..0]}	&	Vector intensity.	\\
\end{tabular} \\
%%

Draws a vector from the current X and Y position to the position specified by
{\tt X[9..0]} and {\tt Y[9..0]}, with the intensity {\tt Z[3..0]}. X and Y are
binary normalised, and the opcode value specifies the number of bits they have
been shifted.

The following table illustrates how the length divisor and bit shift count
relate to the opcode value:

\begin{tabular}{|l|l|l|}
	\hline
	{\bf Opcode}	&	{\bf Divisor}	&	{\bf Bits shifted}	\\ \hline
	09h				&	1				&	0					\\ \hline
	08h				&	2				&	1					\\ \hline
	07h				&	4				&	2					\\ \hline
	06h				&	8				&	3					\\ \hline
	05h				&	16				&	4					\\ \hline
	04h				&	32				&	5					\\ \hline
	03h				&	64				&	6					\\ \hline
	02h				&	128				&	7					\\ \hline
	01h				&	256				&	8					\\ \hline
	00h				&	512				&	9					\\ \hline
\end{tabular}

For example, if a vector is drawn using opcode 09h, it will be drawn at the
exact length specified in the VCTR instruction. If that same vector is drawn
again using opcode 08h, it will be drawn half as long. The same rule applies to
all the other VCTR opcodes.

\vspace{1 cm}
\end{minipage}

%%%%%%%%%%%%%%%%%%%%%%%%%%%%%%%%%%%%%%%%%%%%%%%%%%%%%%%%%%%%%%%%%%%%%%%%%%%%%%%
%% LABS: 0Ah

\begin{minipage}{\textwidth}   \setlength{\parindent}{\savedparindent}
\opcodeentry{LABS}{Load Absolute}{Opcode 0Ah}
%{\bf Encoding:}
\begin{tabular*}{1\textwidth}{@{\extracolsep{\fill}} |c|c|c|c|c|c|c|c|c|c|c|c|c|c|c|c|}
	\multicolumn{16}{ c }{\bf Word 0} \\
	\mcol{15} & \mcol{14} & \mcol{13} & \mcol{12} & \mcol{11} & \mcol{10} & \mcol{9} & \mcol{8} & \mcol{7} & \mcol{6} & \mcol{5} & \mcol{4} & \mcol{3} & \mcol{2} & \mcol{1} & \mcol{0} \\ \hline
	\multicolumn{4}{|c|}{OP[3..0]}	& 0		& Y\tsub{S}	& \multicolumn{10}{c|}{Y[9..0]} \\ \hline
\end{tabular*}
\\
\vspace{2 mm}
\\
\begin{tabular*}{1\textwidth}{@{\extracolsep{\fill}} |c|c|c|c|c|c|c|c|c|c|c|c|c|c|c|c|}
	\multicolumn{16}{ c }{\bf Word 1} \\
	\mcol{15} & \mcol{14} & \mcol{13} & \mcol{12} & \mcol{11} & \mcol{10} & \mcol{9} & \mcol{8} & \mcol{7} & \mcol{6} & \mcol{5} & \mcol{4} & \mcol{3} & \mcol{2} & \mcol{1} & \mcol{0} \\ \hline
	\multicolumn{4}{|c|}{SF[3..0]}	& 0		& X\tsub{S}	& \multicolumn{10}{c|}{X[9..0]} \\ \hline
\end{tabular*} \\ \\
%%
{\bf Operands:}

\begin{tabular}{ l l }
	{\bf OP[3..0]}	&	Opcode nibble.			\\
	{\bf Y\tsub{S}}	&	Y axis sign bit.		\\
	{\bf Y[9..0]}	&	Y axis position.		\\
	{\bf X\tsub{S}}	&	X axis sign bit.		\\
	{\bf X[9..0]}	&	X axis position.		\\
	{\bf SF[3..0]}	&	Global scale factor.	\\
\end{tabular} \\
%%

Sets the X and Y position counters to the values stored in {\tt X[9..0]} and
{\tt Y[9..0]} respectively, and sets the global scale factor to the value
stored in {\tt SF[3:0]}.

The global scale factor is applied to all vectors drawn after the LABS
instruction is executed, and decodes as follows:

\begin{tabular}{|l|l l|}
	\hline
	{\bf Global scale}	&	\multicolumn{2}{l|}{\bf Scale factor}	\\ \hline
	1111b				&	/ 2				&	(divide by 2)		\\ \hline
	1110b				&	/ 4				&	(divide by 4)		\\ \hline
	1101b				&	/ 8				&	(divide by 8)		\\ \hline
	1100b				&	/ 16			&	(divide by 16)		\\ \hline
	1011b				&	/ 32			&	(divide by 32)		\\ \hline
	1010b				&	/ 64			&	(divide by 64)		\\ \hline
	1001b				&	/ 128			&	(divide by 128)		\\ \hline
	1000b				&	/ 256?			&	{\bf (unconfirmed)}	\\ \hline
	0000b				&	Zero?			&	{\bf (unconfirmed)}	\\ \hline
	0001b				&	* 2				&	(multiply by 2)		\\ \hline
	0010b				&	* 4				&	(multiply by 4)		\\ \hline
	0011b				&	* 8				&	(multiply by 8)		\\ \hline
	0100b				&	* 16			&	(multiply by 16)	\\ \hline
	0101b				&	* 32			&	(multiply by 32)	\\ \hline
	0110b				&	* 64			&	(multiply by 64)	\\ \hline
	0111b				&	* 128			&	(multiply by 128)	\\ \hline
\end{tabular}

When a vector is drawn, the global scale factor is used to provide additional
scaling. In the case of Asteroids, this means that only one asteroid image has
to be stored in memory; the global scale factor allows that image to be
expanded or contracted as required.

\vspace{1 cm}
\end{minipage}

%%%%%%%%%%%%%%%%%%%%%%%%%%%%%%%%%%%%%%%%%%%%%%%%%%%%%%%%%%%%%%%%%%%%%%%%%%%%%%%
%% HALT: 0Bh

\begin{minipage}{\textwidth}   \setlength{\parindent}{\savedparindent}
\opcodeentry{HALT}{Halt}{Opcode 0Bh}
%{\bf Encoding:}
\begin{tabular*}{1\textwidth}{@{\extracolsep{\fill}} |c|c|c|c|c|c|c|c|c|c|c|c|c|c|c|c|}
	\multicolumn{16}{ c }{\bf Word 0} \\
	\mcol{15} & \mcol{14} & \mcol{13} & \mcol{12} & \mcol{11} & \mcol{10} & \mcol{9} & \mcol{8} & \mcol{7} & \mcol{6} & \mcol{5} & \mcol{4} & \mcol{3} & \mcol{2} & \mcol{1} & \mcol{0} \\ \hline
	\multicolumn{4}{|c|}{OP[3..0]}	& 0 & 0 & 0 & 0 & 0 & 0 & 0 & 0 & 0 & 0 & 0 & 0 \\ \hline
\end{tabular*} \\ \\
%%
{\bf Operands:}

\begin{tabular}{ l l }
	{\bf OP[3..0]}	&	Opcode nibble.			\\
\end{tabular} \\
%%

Halts the vector generator and blanks the screen.

\vspace{1 cm}
\end{minipage}

%%%%%%%%%%%%%%%%%%%%%%%%%%%%%%%%%%%%%%%%%%%%%%%%%%%%%%%%%%%%%%%%%%%%%%%%%%%%%%%
%% JSRL: 0Ch

\begin{minipage}{\textwidth}   \setlength{\parindent}{\savedparindent}
\opcodeentry{JSRL}{Jump to subroutine}{Opcode 0Ch}
%{\bf Encoding:}
\begin{tabular*}{1\textwidth}{@{\extracolsep{\fill}} |c|c|c|c|c|c|c|c|c|c|c|c|c|c|c|c|}
	\multicolumn{16}{ c }{\bf Word 0} \\
	\mcol{15} & \mcol{14} & \mcol{13} & \mcol{12} & \mcol{11} & \mcol{10} & \mcol{9} & \mcol{8} & \mcol{7} & \mcol{6} & \mcol{5} & \mcol{4} & \mcol{3} & \mcol{2} & \mcol{1} & \mcol{0} \\ \hline
	\multicolumn{4}{|c|}{OP[3..0]}	& \multicolumn{12}{c|}{A[11..0]} \\ \hline
\end{tabular*} \\ \\
%%
{\bf Operands:}

\begin{tabular}{ l l }
	{\bf OP[3..0]}	&	Opcode nibble.			\\
	{\bf A[11..0]}	&	Target address.			\\
\end{tabular} \\
%%

Pushes the current value of the program counter onto the stack and sets the
program counter to the address stored in {\tt A[11..0]}.

\vspace{1 cm}
\end{minipage}

%%%%%%%%%%%%%%%%%%%%%%%%%%%%%%%%%%%%%%%%%%%%%%%%%%%%%%%%%%%%%%%%%%%%%%%%%%%%%%%
%% RTSL: 0Dh

\begin{minipage}{\textwidth}   \setlength{\parindent}{\savedparindent}
\opcodeentry{RTSL}{Return from subroutine}{Opcode 0Dh}
%{\bf Encoding:}
\begin{tabular*}{1\textwidth}{@{\extracolsep{\fill}} |c|c|c|c|c|c|c|c|c|c|c|c|c|c|c|c|}
	\multicolumn{16}{ c }{\bf Word 0} \\
	\mcol{15} & \mcol{14} & \mcol{13} & \mcol{12} & \mcol{11} & \mcol{10} & \mcol{9} & \mcol{8} & \mcol{7} & \mcol{6} & \mcol{5} & \mcol{4} & \mcol{3} & \mcol{2} & \mcol{1} & \mcol{0} \\ \hline
	\multicolumn{4}{|c|}{OP[3..0]}	& 0 & 0 & 0 & 0 & 0 & 0 & 0 & 0 & 0 & 0 & 0 & 0 \\ \hline
\end{tabular*} \\ \\
%%
{\bf Operands:}

\begin{tabular}{ l l }
	{\bf OP[3..0]}	&	Opcode nibble.			\\
\end{tabular} \\
%%

Pulls an address off of the stack and loads it into the program counter.

\vspace{1 cm}
\end{minipage}

%%%%%%%%%%%%%%%%%%%%%%%%%%%%%%%%%%%%%%%%%%%%%%%%%%%%%%%%%%%%%%%%%%%%%%%%%%%%%%%
%% JMPL: 0Eh

\begin{minipage}{\textwidth}   \setlength{\parindent}{\savedparindent}
\opcodeentry{JMPL}{Jump}{Opcode 0Eh}
%{\bf Encoding:}
\begin{tabular*}{1\textwidth}{@{\extracolsep{\fill}} |c|c|c|c|c|c|c|c|c|c|c|c|c|c|c|c|}
	\multicolumn{16}{ c }{\bf Word 0} \\
	\mcol{15} & \mcol{14} & \mcol{13} & \mcol{12} & \mcol{11} & \mcol{10} & \mcol{9} & \mcol{8} & \mcol{7} & \mcol{6} & \mcol{5} & \mcol{4} & \mcol{3} & \mcol{2} & \mcol{1} & \mcol{0} \\ \hline
	\multicolumn{4}{|c|}{OP[3..0]}	& \multicolumn{12}{c|}{A[11..0]} \\ \hline
\end{tabular*} \\ \\
%%
{\bf Operands:}

\begin{tabular}{ l l }
	{\bf OP[3..0]}	&	Opcode nibble.			\\
	{\bf A[11..0]}	&	Target address.			\\
\end{tabular} \\
%%

Sets the program counter to the address stored in {\tt A[11..0]} but, unlike
{\tt JSRL}, does not push the return address on to the stack.

\vspace{1 cm}
\end{minipage}

%%%%%%%%%%%%%%%%%%%%%%%%%%%%%%%%%%%%%%%%%%%%%%%%%%%%%%%%%%%%%%%%%%%%%%%%%%%%%%%
%% SVEC: 0Fh

\begin{minipage}{\textwidth}   \setlength{\parindent}{\savedparindent}
\opcodeentry{SVEC}{Draw Short Vector}{Opcode 0Fh}
%{\bf Encoding:}
\begin{tabular*}{1\textwidth}{@{\extracolsep{\fill}} |c|c|c|c|c|c|c|c|c|c|c|c|c|c|c|c|}
	\multicolumn{16}{ c }{\bf Word 0} \\
	\mcol{15} & \mcol{14} & \mcol{13} & \mcol{12} & \mcol{11} & \mcol{10} & \mcol{9} & \mcol{8} & \mcol{7} & \mcol{6} & \mcol{5} & \mcol{4} & \mcol{3} & \mcol{2} & \mcol{1} & \mcol{0} \\ \hline
	\multicolumn{4}{|c|}{OP[3..0]}	& SF0	&	X\tsub{S}	&	\multicolumn{2}{c}{X[1..0]}	&	\multicolumn{4}{|c|}{Z[3..0]}	&	SF1	&	Y\tsub{S}	&	\multicolumn{2}{c|}{Y[1..0]}	\\ \hline
\end{tabular*} \\ \\
%%
{\bf Operands:}

\begin{tabular}{ l l }
	{\bf OP[3..0]}	&	Opcode nibble.		\\
	{\bf X[1..0]}	&	X delta.			\\
	{\bf X\tsub{S}}	&	X sign.				\\
	{\bf Y[1..0]}	&	X delta.			\\
	{\bf Y\tsub{S}}	&	X sign.				\\
	{\bf Z[3..0]}	&	Vector intensity.	\\
	{\bf SF[1..0]}	&	Scale factor.		\\
\end{tabular} \\
%%

Draws a vector from the current X and Y position to the position specified by
{\tt X[1..0]} and {\tt Y[1..0]}, with the intensity {\tt Z[3..0]}. This is a
variant of the VCTR instruction that is usually used for drawing very small
vectors where absolute accuracy is not required, for example text or numbers.

This instruction performs the same function as a VCTR instruction, but in a
more compact (and slightly faster) way. {\tt X[1..0]} and {\tt Y[1..0]} in an
SVEC instruction only contain bits 8 and 9 (the two least significant bits of
the high byte) of the X and Y addresses. The scale factor decodes as follows:

\begin{tabular}{|l|l|l|}
	\hline
	{\bf Scale bits}&	{\bf Divisor}	&	{\bf Bits shifted}	\\ \hline
	00b	(00d)		&	128				&	7					\\ \hline
	01b	(01d)		&	64				&	6					\\ \hline
	10b	(02d)		&	32				&	5					\\ \hline
	11b	(03d)		&	16				&	4					\\ \hline
\end{tabular}

This means that the input coordinate values and scale factors decode to the
following vector lengths:

\begin{tabular}{|l|c|c|c|c|}
	\hline
	\multicolumn{1}{|c|}{\bf Scale}	&	\multicolumn{4}{c|}{\bf Input}						\\ \cline{2-5}
							&	{\bf 00b}	&	{\bf 01b}	&	{\bf 10b}	&	{\bf 11b}	\\ \hline
	00b (divide by 128)		&	{\bf 0}		&	{\bf 2}		&	{\bf 4}		&	{\bf 6}		\\ \hline
	01b (divide by 64)		&	{\bf 0}		&	{\bf 4}		&	{\bf 8}		&	{\bf 12}	\\ \hline
	10b (divide by 32)		&	{\bf 0}		&	{\bf 8}		&	{\bf 16}	&	{\bf 24}	\\ \hline
	11b (divide by 16)		&	{\bf 0}		&	{\bf 16}	&	{\bf 32}	&	{\bf 48}	\\ \hline
\end{tabular}


\vspace{1 cm}
\end{minipage}

%%%%%%%%%%%%%%%%%%%%%%%%%%%%%%%%%%%%%%%%%%%%%%%%%%%%%%%%%%%%%%%%%%%%%%%%%%%%%%%
%%%%%%%%%%%%%%%%%%%%%%%%%%%%%%%%%%%%%%%%%%%%%%%%%%%%%%%%%%%%%%%%%%%%%%%%%%%%%%%
% Restore parindent
\setlength{\parindent}{\savedparindent}
