\section{Introduction}
Over the past few years, interest in the emulation of video games has been
increasing rapidly. Modern computer hardware has allowed arcade games like
Asteroids and Star Wars to be accurately simulated on a commonly-available
home computer system. Although Asteroids was one of the first games to be
emulated, very little documentation exists on the hardware, beyond the memory
maps and notes in the MAME source code.

One of the more unusual parts of the Asteroids hardware is the Digital Vector
Generator, shown on Sheet 2A of the Asteroids schematic set. Outside of Atari,
very little documentation exists on it. In this article, I will explain how
the DVG works, and how to write code for it.

This article is intended to supplement the notes included in the Asteroids
schematics, not replace them. It also concentrates more on the programming side
of the DVG, rather than attempting to explain the complexities of the DVG
circuitry.

\section{Thanks and Acknowledgements}
Thanks are due to the following people, who assisted (either directly or
indirectly) in the creation of this document:

\begin{itemize}
	\item{{\bf Jed Margolin} (\url{http://www.jmargolin.com}) --- Wrote the
article ``The Secret Lives of Vector Generators'', which explains some of the
inner workings of the Atari AVG and DVG. He also explained (in an e-mail) how
the DVG represents coordinates.}

 	\item{{\bf Howard Delman} (\url{http://www.rawbw.com/~delman/}) --- Designer of
the DVG. Answered a few technical questions on the DVG hardware.}

 	\item{{\bf Chris Pile} (\url{http://web.archive.org/*/http://members.tripod.com/asteroids/}) --- Documented
part of the DVG, and published his notes online (filename: {\tt roidinfo.zip}).
Note that the original site has ceased to exist, hence the link to the Web
Archive mirror of the same page.}

 	\item{{\bf Eric Smith} (\url{http://www.brouhaha.com/~eric/} --- Released
the source code to the VECSIM emulator, which was used to get a basic overview
of the DVG instruction set.}
\end{itemize}

\subsection{Change log}
\begin{itemize}
  \item{{\bf 18 September 2006}: Added documentation on the '00h' instruction opcode.}
\end{itemize}
