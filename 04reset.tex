\section{Reset Sequence}
When the DVG's \oline{RESET} input goes active, the following actions are
performed by the DVG circuitry at roughly the same time:

\begin{itemize}
	\item{Instruction latch {\tt F7} and Y offset latch {\tt H7} are cleared to zero}
	\item{State flip-flop {\tt D8} is cleared to zero}
	\item{{\tt HALT} flip-flop {\tt A9} is set; {\tt HALT} and {\tt \oline{HALT}} go active}
\end{itemize}

The DVG will hold in state 0 until the 6502 asserts the {\tt \oline{DMAGO}}
line. This resets the {\tt HALT} flip-flop, causing {\tt HALT} and
{\tt \oline{HALT}} to be deasserted, thus releasing the halt state and allowing
the state machine to begin executing opcodes from the vector instruction memory.

Once the DVG is released from the halt state, it will execute a {\tt DMALD}
micro-instruction, which loads the program counter from {\tt DVY[11..0]}. As
DVY has been cleared to zero during reset, the program counter will be set to
zero.

After the program counter has been reset, the state machine will load the
instruction latch, then the {\tt DVY} latch. The program counter will be
incremented after the {\tt DVY} latch has been loaded. If an opcode for a
two-word instruction has been loaded into the instruction latch, the {\tt SCALE}
and {\tt DVX} latches will also be loaded from the vector program memory, and
the program counter will be incremented again.

% >{\raggedleft} for right align
% >{\centering} for center align
%\begin{tabularx}{\linewidth}{|c|c|c|X|}
%	\hline
%	{\bf State}		&	{\bf Next state}		&	{\bf Signal}		&	{\bf Action}				\tabularnewline	\hline
%	$0$				&	$9$						&	DMALD				&	Load PC with DVY			\tabularnewline	\hline
%\end{tabularx}

%% TODO: reference state diagram, and maybe create state diagram?
