\section{Coordinate System}
Absolute coordinates in DVG terms are expressed as binary numbers from 0 to
1023 in the form {\tt (x, y)}, where {\tt (0, 0)} is the top-left of the screen
area, {\tt (511, 511)} is the centre of the screen, and {\tt (1023, 1023)} is
the bottom-right of the screen area.

Relative coordinates are expressed in sign-magnitude form, and are processed
using binary normalisation to ensure that the vectors they represent are drawn
as fast as possible. Sign-magnitude form stores the absolute value of the
number in the least significant bits, and the sign in the most significant bit.
The sign bit on the DVG is high when a number is negative. As a consequence of
this, there are two ways to represent zero. Relative coordinates are used by
all the vector drawing instructions.

Absolute coordinates are expressed in two's complement form. When the most
significant bit is set, the number is taken to be negative. To convert a
negative number back into its absolute value, invert all the bits in the word
and add one to the result. Absolute coordinates are only used by the {\tt LABS}
instruction.
